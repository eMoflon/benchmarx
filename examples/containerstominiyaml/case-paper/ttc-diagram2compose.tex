%%
\documentclass[twocolumn]{ceurart}
\usepackage[utf8]{inputenc}
\usepackage{listings}

\graphicspath{{figs/}{}}

%%
%% \BibTeX command to typeset BibTeX logo in the docs
\AtBeginDocument{%
  \providecommand\BibTeX{{%
    Bib\TeX}}}

\sloppy{}

%%
%% Rights management information.
%% CC-BY is default license.
\copyrightyear{2023}
\copyrightclause{Copyright for this paper by its authors.
  Use permitted under Creative Commons License Attribution 4.0
  International (CC BY 4.0).}

%%
%% This command is for the conference information
\conference{TTC'23: 15th Transformation Tool Contest, 
  Part of the Software Technologies: Applications and Foundations (STAF) federated conferences,
  Eds. A. Boronat, A. Garc\'{i}a-Dom\'{i}nguez, and G. Hinkel,
  20 July 2023, Leicester, UK.}

% From https://tex.stackexchange.com/questions/152829/

\newcommand\YAMLcolonstyle{\color{red}\mdseries}
\newcommand\YAMLkeystyle{\color{black}\bfseries}
\newcommand\YAMLvaluestyle{\color{blue}\mdseries}

\makeatletter

% here is a macro expanding to the name of the language
% (handy if you decide to change it further down the road)
\newcommand\language@yaml{yaml}

\expandafter\expandafter\expandafter\lstdefinelanguage
\expandafter{\language@yaml}
{
  keywords={true,false,null,y,n},
  keywordstyle=\color{darkgray}\bfseries,
  basicstyle=\YAMLkeystyle,                                 % assuming a key comes first
  sensitive=false,
  comment=[l]{\#},
  morecomment=[s]{/*}{*/},
  commentstyle=\color{purple}\ttfamily,
  stringstyle=\YAMLvaluestyle\ttfamily,
  moredelim=[l][\color{orange}]{\&},
  moredelim=[l][\color{magenta}]{*},
  moredelim=**[il][\YAMLcolonstyle{:}\YAMLvaluestyle]{:},   % switch to value style at :
  morestring=[b]',
  morestring=[b]",
  literate =    {---}{{\ProcessThreeDashes}}3
                {>}{{\textcolor{red}\textgreater}}1     
                {|}{{\textcolor{red}\textbar}}1 
                {\ -\ }{{\mdseries\ -\ }}3,
}

% switch to key style at EOL
\lst@AddToHook{EveryLine}{\ifx\lst@language\language@yaml\YAMLkeystyle\fi}
\makeatother

\newcommand\ProcessThreeDashes{\llap{\color{cyan}\mdseries-{-}-}}


%%
%% The "title" command has an optional parameter,
%% allowing the author to define a "short title" to be used in page headers.
\title{Asymmetric and Directed Bidirectional Transformation for Container Orchestrations}

%%
%% The "author" command and its associated commands are used to define
%% the authors and their affiliations.
%% Of note is the shared affiliation of the first two authors, and the
%% "authornote" and "authornotemark" commands
%% used to denote shared contribution to the research.
\author[1]{Antonio Garcia-Dominguez}[
 email=a.garcia-dominguez@york.ac.uk,
 orcid=0000-0002-4744-9150,
]
\address[1]{University of York, York, UK}


%%
%% The abstract is a short summary of the work to be presented in the
%% article.
\begin{abstract}
  In many DevOps scenarios, tools operate from declarative models of intended
  system configuration (e.g. Ansible/Puppet/Chef descriptions of
  infrastructure-as-code, or Kubernetes and Docker Compose descriptions of
  orchestrations of containers). DevOps-oriented domain-specific modeling
  notations will typically only cover a subset of all the capabilities in these
  configuration formats: this means users will need to manually edit the
  configuration files generated from the higher-level models. In many editing
  sessions, users will also touch upon parts that came from the high-level
  model, and will want that high-level model to be updated accordingly.
  Likewise, a user may want to introduce a change through the high-level model
  and not lose the YAML customisations that are unrelated to the high-level
  model. These requirements imply a need for a bidirectional transformation
  (``bx'') which is asymmetric (the configuration file contains all the
  information in the high-level model and more), and directed (changes are only
  applied to one side at a time). This case proposes revisiting the current
  state of bx tools for asymmetric and directed transformations, and complements
  the prior Families to Persons case from TTC 2017, which focused on a
  symmetrical and directed transformation. The case will reuse the Benchmarx
  framework from the TTC 2017 case.
\end{abstract}

%%
%% end of the preamble, start of the body of the document source.
\begin{document}

%%
%% Keywords. The author(s) should pick words that accurately describe
%% the work being presented. Separate the keywords with commas.
\begin{keywords}
container orchestration \sep
bidirectional transformations \sep
model merging \sep
graphical models \sep
YAML
\end{keywords}


%%
%% This command processes the author and affiliation and title
%% information and builds the first part of the formatted document.
\maketitle

\section{Introduction}

DevOps was defined by Leite et al.~\cite{leite_survey_2020} as a ``collaborative
and multidisciplinary effort within an organization to automate continuous
delivery of new software versions, while guaranteeing their correctness and
reliability''. The rising interest in DevOps (with over 10\% of the 61,302
responses to the Stack Overflow 2022 Developer
Survey\footnote{\url{https://survey.stackoverflow.co/2022/}} considering
themselves ``DevOps specialists'') has motivated the creation of a number of
domain-specific modelling notations for it, covering aspects such as
microservice architectures~\cite{sorgalla_applying_2021}, DevOps
processes~\cite{colantoni_devopsml_2020}, or multi-cloud
applications~\cite{ferry_cloudmf_2018}.

At a technical level, the automated continuous delivery efforts in DevOps
typically require using tools to automate deployment. These include
infrastructure-as-code tools (e.g.
Puppet\footnote{\url{https://www.puppet.com/}} or
Ansible\footnote{\url{https://www.ansible.com/}}), and container orchestration
tools such as Kubernetes\footnote{\url{https://kubernetes.io/}} or Docker
Compose\footnote{\url{https://docs.docker.com/compose/}}. Many of these tools
operate by reading a declarative description of the desired system state or the
intended combination of containers, usually written in a structured format
(e.g.\ YAML\footnote{\url{https://yaml.org/}}) according to a loosely defined
schema (c.f.\ the Docker Compose file format reference, which evolves from
version to
version\footnote{\url{https://docs.docker.com/compose/compose-file/}}).

It stands to reason that DevOps model-driven approaches would often aim to
generate at least some of these configuration files from the high-level
descriptions of the intended service compositions. Piedade et al.\ observed a
significant reduction in development effort with a visual notation for
developing Docker Compose container orchestrations~\cite{piedade_visual_2022},
while also noticing that several of the existing visual tools for Docker Compose
lacked support for certain Docker Compose concepts (e.g.\ DockStation did not
support specifying networks). Their high-level descriptions will only model the
subset of the capabilities of the underlying tools that is relevant for their
abstractions, as trying to capture all capabilities would overcomplicate the
models and make them more brittle to minor changes in the underlying
configuration file formats. From this limitation, it follows that users would
typically manually customise the generated configuration files to cover the
aspects not described by the high-level model. Users may later want to update
the high-level model from the configuration file, to use it for visualisation
(e.g.\ for onboarding new developers) or for reorganising the system in a more
approachable notation with domain-specific validation rules.

It is worth noting that there are some agreed-upon specifications in cloud
computing that have been adapted into model-driven approaches. Zalila et
al.~\cite{zalila_model-driven_2019} proposed OCCIware Studio, a model-driven
toolchain that formalises the concepts in the Open Cloud Computing Interface
(OCCI, a unified RESTful protocol for cloud computing management) into an
OCCIware Ecore metamodel, and provides a runtime component for design,
deployment, execution, and supervision of cloud applications. Challita et
al.\ later proposed TOSCA Studio~\cite{challita_model-based_2021}, also based on
OCCIware, which provides a model-driven approach to design OASIS Topology and
Orchestration Specification for Cloud Applications (TOSCA) descriptions: these
are usually written in YAML and only describe the structure of cloud
applications in a declarative manner, leaving the exact implementation up to the
TOSCA-supporting cloud provider.
OpenTOSCA\footnote{\url{https://www.opentosca.org/}} is an open-source
end-to-end toolchain for deploying and managing cloud applications, which
includes Eclipse Winery, a web-based environment for visual modeling of TOSCA
cloud application topologies (which generates TOSCA YAML descriptions).

This paper proposes a case based on a scenario inspired by the findings of
Piedade et al.~\cite{piedade_visual_2022}, focusing on container orchestration
with Docker Compose. A high-level graphical domain-specific model (implemented
with Sirius) is transformed into a Docker Compose YAML file, which can be
customised by the user using a plain text editor. The high-level model can be
updated from the YAML file at any time. It should also be possible to edit the
high-level model and push the changes to the YAML file, while retaining any
elements that were not part of the high-level graphical DSML.

At an essential level, this case implies the definition of a bidirectional
transformation (``bx'' from now on) between the high-level DSML and Docker
Compose YAML files. In TTC 2017, the Families to Persons case by Anjorin et
al.~\cite{anjorin_families_2017} evaluated the available approaches for
symmetric and directed bx using the proposed Benchmarx framework. This work was
later updated and expanded upon in a journal
paper~\cite{anjorin_benchmarking_2020}, which also collected a number of useful
terms to describe bx, as well as a feature model to cover the variability of bx
tools. Families to Persons was symmetric (neither side was a view of the other,
with information loss happening in both directions), and directed
(consistency-relevant changes were only applied to one side at a time). The
proposed case is still directed, but it is asymmetrical (the Docker Compose YAML
file contains strictly more information than the high-level model, so
information loss only happens in one direction). While this should make it
conceptually ``easier'' than the symmetric Families to Persons bx, the mapping
is also more complicated, with some objects in the high-level model being turned
into simple string concatenations in the target model. At the same time, it can
be argued that the generation of Docker Compose YAML files is a more
industrially relevant scenario: if the current state of the art in bx tools
(which may have significantly evolved since the TTC 2017 case) can handle it
well, this could prove to be an interesting application niche.

\section{Modeling Languages}

\begin{figure}
  \centering
  \includegraphics[width=\columnwidth]{containers-metamodel}
  \caption{Class diagram for the Containers metamodel}%
  \label{fig:containers-metamodel}
\end{figure}

\begin{figure}
  \centering
  \includegraphics[width=\columnwidth]{miniyaml-metamodel}
  \caption{Class diagram for the MiniYAML metamodel}%
  \label{fig:miniyaml-metamodel}
\end{figure}

The proposed bx is between two languages: a ``Containers'' domain-specific
modelling language (shown in Figure~\ref{fig:containers-metamodel}), and a
simplification of the YAML data model called ``MiniYAML'' (shown in
Figure~\ref{fig:miniyaml-metamodel}).

\subsection{Abstract syntax}
\label{sec:abstract-syntax}

\newcommand*{\metaclass}[1]{\textsc{#1}}
\newcommand*{\feature}[1]{\texttt{#1}}

Models conforming to the Containers metamodel
(Figure~\ref{fig:containers-metamodel}) have a \metaclass{Composition} as their
root object, containing a number of \metaclass{Node}s of various types. An
\metaclass{Image} represents a specific Docker image by its full name including
the registry (if it is not the Docker Hub) and tag, as stored in its
\feature{image} attribute. A \metaclass{Container} is a component that runs one
or more \feature{replicas} of a certain \metaclass{Image}. A
\metaclass{Container} may have \metaclass{Volume\-Mount}s of certain
\metaclass{Volume}s (units of persistent storage) at specific \feature{path}s.
\metaclass{Container}s and \metaclass{Volume}s are \metaclass{NamedElement}s,
which have a \feature{name} that also acts as their unique identifier. A
\metaclass{Container} may \feature{dependOn} other \metaclass{Container}s,
meaning that it should only be started after its dependencies have been started.

On the other hand, a model conforming to the MiniYAML metamodel
(Figure~\ref{fig:miniyaml-metamodel}) has a \metaclass{Map} as its root object,
which contains \metaclass{MapEntry} objects. Each \metaclass{MapEntry} has a
\feature{key} (a string, which should be unique within its containing
\metaclass{Map}), and a \feature{value}. Besides \metaclass{Map}, other types of
\metaclass{Value}s include \metaclass{List}s (of \metaclass{Value}s), and
\metaclass{Scalar} values with a string (this is a simplification from YAML,
which can support integer and floating point types through its JSON schema).

\subsection{Concrete syntax}
\label{sec:concrete-syntax}

\newcommand*{\eclipseproject}[1]{\texttt{#1}}
\newcommand*{\javaclass}[1]{\textsc{#1}}

\begin{figure*}
  \centering
  \includegraphics[width=\textwidth]{containers-model}
  \caption{Example containers model, based on the AutoFeedback open-source system}%
  \label{fig:containers-model}
\end{figure*}

The concrete syntax of the Containers modelling language is implemented through
Eclipse Sirius\footnote{\url{https://www.eclipse.org/sirius/}} and exemplified
in Figure~\ref{fig:containers-model}, which models the container orchestration
used by the AutoFeedback system developed by the
author\footnote{\url{https://gitlab.com/autofeedback/autofeedback-webapp/-/blob/master/docker-compose.yml}}.
\metaclass{Container}s are grey rectangles decorated with a puzzle piece icon,
labelled after their \feature{name}. A \metaclass{Container} may contain yellow
ovals representing their \metaclass{Volume\-Mount}s, labelled after their
\feature{path}s and decorated with a folder icon. An \metaclass{Image} is
reflected as a green oval with a cardboard box icon, labelled after their
\feature{image}. \metaclass{Volume}s are purple rectangles with a hard disk
icon, labelled after their \feature{name}. Note that the \feature{replicas} of a
\metaclass{Container} is not part of its graphical syntax, but can be edited
through the Properties view of the Sirius editor.

\lstinputlisting[language=yaml,columns=flexible,float,caption={Example YAML from Figure~\ref{fig:containers-model}},label=lst:yaml-example,frame=tb]{listings/MyContainers.yml}

The MiniYAML language does not have an explicitly defined concrete syntax: while
the case artifact includes a tree-based editor autogenerated from the metamodel,
the ultimate concrete syntax is YAML itself. The case artifact includes an
\eclipseproject{uk.ac.york.ttc.mini\-yaml.model\-2yaml} project with a
\javaclass{MiniYAMLConverter} Java class which uses
SnakeYAML\footnote{\url{https://bitbucket.org/snakeyaml/snakeyaml}} to
automatically convert between Mini\-YAML models in XMI format, and YAML files.

\section{Intended Transformations}

The general intent of the transformation is to start from a model as the one in
Figure~\ref{fig:containers-model}, and produce a YAML document such as the one
in Listing~\ref{lst:yaml-example}. This YAML document can be edited manually in
various ways: a user could do a find-and-replace to rename a given container, or
they could add extra options for a given container or volume which are not part
of the Containers metamodel. It should be possible for the user to update the
Containers model from the YAML file at any point. It should also be possible to
edit a Containers model and update the YAML file from it, while keeping any
customisations that are unrelated to the Containers metamodel.

\subsection{High-level description}

In its forward direction (from Containers to MiniYAML), operating in batch mode
(where the MiniYAML model does not exist yet), the transformation should follow
these rules:

\begin{enumerate}
\item A \metaclass{Composition} should be transformed into a \metaclass{Map}
  with three keys: \feature{version} set to a ``2.4'' \metaclass{Scalar},
  \feature{services} set to a \metaclass{Map} whose \metaclass{MapEntry} objects
  are produced from the \metaclass{Container}s, and \feature{volumes} set to a
  \metaclass{Map} produced from the \metaclass{Volume}s.

\item A \metaclass{Container} should be transformed into a \metaclass{MapEntry}
  where the \feature{key} is equal to its \feature{name}. The value of the
  \metaclass{MapEntry} should be a \metaclass{Map} of its own, with at least the
  \feature{image} key set to the \feature{image} of the \metaclass{Image} of the
  \metaclass{Container}.

  The \metaclass{Map} may also have keys for:
  \begin{itemize}
  \item \feature{replicas}, if the value is different from 1.
  \item \feature{volumes}, set to a \metaclass{List} produced from the
    \metaclass{VolumeMount}s of the \metaclass{Container}.
  \item \feature{depends\_on}, set to a \metaclass{List} of \metaclass{Scalar}s
    with the names of the \metaclass{Container}s that this \metaclass{Container}
    depends upon.
  \end{itemize}

\item A \metaclass{VolumeMount} should be transformed into a \metaclass{Scalar}
  whose value should be of the form ``volumeName:path''.

\item A \metaclass{Volume} should be turned into a \metaclass{MapEntry} whose
  \feature{key} should be its \feature{name}. The \metaclass{MapEntry} should
  not have a value.
\end{enumerate}

If the MiniYAML model already exists before running the transformation forward,
then the containers, volumes, volume mounts, replicas, and inter-container
dependencies of the Containers model should replace those of the MiniYAML model,
while preserving any other elements outside the Containers metamodel (e.g.\ a
custom \feature{restart} entry in a container's \metaclass{Map}). At the very
least, adding or removing one of these elements from the Containers model should
add or remove the relevant element in the MiniYAML model. Ideally, the
transformation should be able to handle the renaming of a \metaclass{Container}
or \metaclass{Volume} while preserving the additional content that is unrelated
to the Containers metamodel. Furthermore, the transformation should minimise
unnecessary changes in the YAML file (e.g.\ changes in the order of the map
entries).

In its backward direction (from MiniYAML to Containers) in batch mode, the
transformation should recover the \metaclass{Composition}s, \metaclass{Image}s,
\metaclass{Volume}s and \metaclass{VolumeMount}s from the same MiniYAML elements
that would have been produced in the forward direction. These will replace the
contents of the Containers model entirely. Ideally, the transformation should
minimise unnecessary changes (e.g.\ changing the path of an \metaclass{Image} in
the model, which would cause unnecessary changes in the Sirius diagrams).

\subsection{Reference implementation}

\newcommand*{\file}[1]{\texttt{#1}}

Besides the above high-level description, the case materials\footnote{\url{https://github.com/agarciadom/benchmarx/tree/main/examples/containerstominiyaml}} include EMF-based
implementations of the Containers and MiniYAML metamodels, and a reference
implementation of the transformation using a combination of languages from the
Eclipse Epsilon open-source project:

\begin{itemize}
\item An ETL (Epsilon Transformation Language) script transforming Containers
  models to MiniYAML models (\file{con\-tai\-ners\-2\-miniyaml.etl}).

\item An ETL script transforming MiniYAML models to Containers models
  (\file{miniyaml2containers.etl}).

\item A combination of an Epsilon Merging Language (EML) script, an Epsilon
  Comparison Language (ECL) script, and an ETL script which can merge two
  MiniYAML models together (\file{mergeMiniyaml.eml},
  \file{compareMiniyaml.ecl}, and \file{mergeMiniyaml.etl}) respectively.

  In this transformation, the ``left'' MiniYAML model is the ``prioritary'' one:
  its containers, volumes, volume mounts, replicas, and inter-container
  dependencies will take precedence over those of the ``right'' MiniYAML model.
  Any other content (e.g.\ customisations outside the Containers metamodel) will
  be merged.

  At a high-level, the ECL script computes a match between the ``left'' and
  ``right'' models based on name-based paths
  (e.g.\ \feature{services.redis.image}), where \metaclass{Scalar}s also
  consider their value. The EML script merges matching elements together, and
  the ETL script copies non-matching elements from either side.
\end{itemize}

These transformations are then encapsulated as Java classes:

\begin{itemize}
\item \javaclass{ContainersToMiniYAML} implements the batch forward
  transformation, \javaclass{MergingContainersToMiniYAML} implements the forward
  transformation with merging if the MiniYAML model already exists, and
  \javaclass{MergingContainersToYAML} class implements the forward
  transformation with merging if the YAML file already exists.

\item \javaclass{MiniYAMLToContainers} implements the batch backward
  transformation from a MiniYAML model to a Containers model, and
  \javaclass{YAMLToContainers} also transforms the YAML file into a MiniYAML
  model before transforming it into a Containers model. The reference
  implementation does not have a ``merging'' version of the backward
  transformation: it replaces the Containers model if it exists.

\end{itemize}

\section{Research questions}

The aim of this case is to explore the capabilities of the current state of the
art of transformation tools in an asymmetric and directed bx. Specifically, the
case is intended to answer these questions:

\begin{enumerate}
\item How concisely can we specify such a bx with current tools?

  Having to maintain separate one-way transformations as in the reference
  implementation would incur significant cost when scaling up to the full
  complexity of real-world metamodels. Ideally, it should be possible to
  implement the bx through a single set of relationships, without repetition.
  This could be done through explicit consistency relationships, through triple
  graph grammars, or through static analysis of a one-way transformation (with
  perhaps some use of heuristics).

\item How well can such a bx preserve customisations in the YAML which are
  outside of the bx, across various types of changes in the models?

  The reference implementation can handle well the case where elements are added
  and removed, but it cannot handle renames well: renaming a container in the
  Containers model will result in losing the additional content in the YAML
  file. A bx tool that can operate with operational deltas (``o-deltas'') would
  most likely be able to handle this case in a more robust manner.

\item How would such a bx scale to larger models, with more containers, more
  volumes, and more custom YAML elements outside of the transformation's
  control?

  In the reference implementation, the merging process of the MiniYAML model
  newly created from the Containers model with the previously existing (and
  potentially customised) MiniYAML model requires pairwise object matching, with
  $O(n^2)$ path comparisons per type. Is such a cost unavoidable, or are there
  more efficient ways to establish and maintain the relationships between the
  Containers and MiniYAML models?
\end{enumerate}

In practice, it is unlikely that the YAML documents will grow particularly
large\footnote{The average size of the \file{composer.yaml} files in the
\file{docker/awesome-compose} Github project is 609B:
\url{https://github.com/docker/awesome-compose.}}. Performance would likely not
be an issue for this bidirectional transformation. Instead, maintainability and
keeping to the principle of ``least change'' would be the most important aspects
to tackle. Still, the case materials include an experiment for evaluating the
scalability of the solutions to larger models.

\section{Evaluation criteria}

Solutions will be evaluated across the following criteria:

\begin{lstlisting}[language=Java,float,caption={Sample code for measuring AST/ASG side of transformation rules modelled in EMF},label=lst:ast-size-emf,columns=fullflexible,frame=tb]
public int countNodes(Resource resource) {
  final TreeIterator<EObject> it = resource.getAllContents();
  int size = 0;
  while (it.hasNext()) {
    it.next();
    ++size;
  }
  return size;
}
\end{lstlisting}

\begin{enumerate}
\item \emph{Correctness}: following the approach from the authors of the
  Benchmarx benchmark~\cite{anjorin_benchmarking_2020}, test cases will check
  that the dependent model is consistent with the master model. This means that
  they should have the same containers, volumes, volume mounts, and images.

  This criteria will be measured according to the \% of test cases that are
  passed. The test cases will cover various scenarios, e.g.\ an initial
  ``batch'' execution in either direction, or the update of the dependent side
  after a certain change in the master side.
  
\item \emph{Conciseness}: a more concise description of the transformation
  should in principle be more maintainable. Since the statement structure can be
  significantly different across languages, the metric will be ``number of nodes
  in the transformation's abstract syntax, ignoring comments''. This differs
  from the approach that was followed in the Benchmarx ``Families to Persons''
  study that is the base of this case~\cite{anjorin_families_2017}, which
  counted words while ignoring comments. This is to accommodate both textual and
  graphical transformation notations (e.g.\ triple graph grammars). For
  instance, if the transformation was implemented as an Eclipse Modeling
  Framework (EMF) model, the metric would be equivalent to the code in
  Listing~\ref{lst:ast-size-emf}.

  The case includes an \file{ast-counter} Maven project which can count the
  number of AST nodes in Java code and in the Epsilon languages used for the
  reference solution. Participants are encouraged to extend this project to
  measure their source languages (e.g.\ by counting the number of elements in an
  XMI-serialised model), by adding the relevant implementations of the
  \javaclass{IFileMeasurer} interface and associating it to the appropriate
  extension inside the static block of the \javaclass{FolderMeasurer} class. It
  is also acceptable to produce these AST measurements separately as part of
  their solution (e.g.\ if no JVM-friendly parsers exist for a transformation
  language).

  The reference implementation includes an Eclipse launch configuration that
  measures the number of AST nodes in its Java and Epsilon source code.
  Participants are encouraged to duplicate this launch configuration for their
  own solutions, providing it with the root folder of their transformation
  source code.

  Note that the reference implementation also includes a \file{count-words.sh}
  script which uses the C preprocessor to remove comments for Epsilon / Java
  programs. This is only to emulate what the old approach (based on words) would
  have produced, for the sake of comparison: it will not be used for the
  contest, as results may not be directly comparable. As an example, these are
  the results of the two measurement methods at the time of writing for the
  reference implementation:

  \begin{itemize}
  \item \emph{AST node counting}: 86 nodes in ECL, 241 nodes in EML, 92 nodes in
    EOL, 805 nodes in ETL, and 1772 nodes in Java, for a total of 2996 nodes.

  \item \emph{Word counting}: 84 words in ECL, 162 words in EML, 81 words in
    EOL, 485 words in ETL, and 809 words in Java, for a total of 1621 words.
  \end{itemize}

\item \emph{Least Change}: beyond just correctness, the transformations should
  avoid making any unnecessary changes that do not impact the consistency of the
  master and dependent model. For instance, in the forward direction, they
  should preserve the additional information in the existing YAML file, and the
  relative order of the keys in the YAML document. In the backward direction,
  they should also preserve the locations of the various nodes, avoiding
  disturbing existing Sirius diagrams whenever possible.

  To measure this, the test suite has been designed to be run in two modes: 1)
  requiring that if the YAML model already exists, the relative order of map
  entries and list items is preserved, and 2) waiving this requirement. Mode 1
  is intended for evaluating ``least change'' (in terms of \% of tests passed in
  this mode), whereas Mode 2 is for evaluating general correctness of the
  transformation.

\item \emph{Scalability}: the transformations should be able to scale to models
  with increasing numbers of containers, volumes, and images. The case materials
  include a \javaclass{ScalabilityMeasurements} class to measure this in the
  forward and backward directions, both in batch and in incremental situations.

\end{enumerate}

% Restructure repo to be a fork of https://github.com/eMoflon/benchmarx

\section{Target prizes}

The prizes will be based on a combination of the three criteria above. The
``Most Complete'' prize will go to the solution that passes the most tests
(resolving ties using the ``Least Change'' criterion). The ``Most Concise''
prize will go to the solution that requires the least nodes, while still passing
the correctness tests for adding and deleting elements (the tests for renaming
elements will not be considered). The ``Most Scalable'' prize will go to the
solution with the lowest execution times, which is still correct in the batch
scenarios and in the incremental addition and removal of containers, volumes,
volume mounts, and images.

If there are enough solutions, an overall ranking can be devised by adding their
rankings in each category, and sorting in ascending order. Ties will be resolved
by sorting in ascending order of standard deviation (therefore, a tool that is
2nd/2nd would be ranked above a tool that is 1st/3rd). Further ties will be
resolved by the case author and TTC organizers.

\section{Journal-quality solution criteria}

To be eligible for a follow-up journal publication, a solution must be correct
in the ``batch'' context in both directions, and in the ``incremental'' context
in regard to addition and removal of containers, volumes, volume mounts, and
images. Conciseness, ``least change'', and scalability are desirable properties,
but not required for such a publication. Ideally, declarative solutions that
support maintainability by not requiring the specification of both
transformation directions would be preferred.

\bibliography{bibliography}

\end{document}
